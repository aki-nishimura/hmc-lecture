\documentclass[18pt]{beamer}

\usetheme{Madrid} % [secheader]
\setbeamertemplate{footline}{}
\usefonttheme{professionalfonts}
\usepackage[english]{babel}
\usepackage{graphicx}
\usepackage[normalem]{ulem}
\usepackage{hyperref}
\usepackage{bm}
\usepackage{dsfont}
\usepackage{subcaption}
\usepackage{natbib}
\usepackage{bbm}
\usepackage{calc}  % For widthof command
\usepackage{xcolor}
\usepackage{natbib}
\bibliographystyle{abbrvnat}
\setcitestyle{authoryear, open={(}, close={)}}

\setbeamerfont{itemize/enumerate subbody}{size=\normalsize} %to set the body size
%\setbeamertemplate{frametitle}[default][center]
\setbeamertemplate{itemize subitem}{\normalsize\raise1.25pt\hbox{\donotcoloroutermaths$\blacktriangleright$}}  %to set the symbol size

% HMC related macros
\newcommand{\position}{\theta}
\newcommand{\momentum}{p}
\newcommand{\bposition}{\bm{\position}}
\newcommand{\bmomentum}{\bm{\momentum}}
\newcommand{\Position}{\Theta}
\newcommand{\Momentum}{P}
\newcommand{\integrationTime}{\tau}

% Math macros
\newcommand{\transpose}{\text{\raisebox{.5ex}{$\intercal$}}}
\newcommand{\diff}{{\rm d}}

% Probability / statistics macros
\newcommand{\given}{\, | \,}
\newcommand{\normal}{\mathcal{N}}

% Variable / Greek symbol macros
\newcommand{\y}{\bm{y}}
\newcommand{\x}{\bm{x}}
\newcommand{\I}{\bm{I}}
\newcommand{\bPhi}{\bm{\Phi}}
\newcommand{\bSigma}{\bm{\Sigma}}
\newcommand{\nparam}{d}

% Utility macros
\newcommand{\todo}[1]{\textcolor{red}{#1}}

% Set the color theme of the presentation
\definecolor{uclablue}{RGB}{39, 116, 174}
\usecolortheme[named=uclablue]{structure}
\colorlet{textcolor}{white}
\setbeamercolor{frametitle}{fg=textcolor}
\setbeamercolor{titlelike}{fg=textcolor}
\setbeamercolor{section in head/foot}{fg=textcolor}
%\colorlet{toccolor}{black}
%\setbeamercolor{section in toc}{fg=toccolor}

%  Block colors
\setbeamercolor{block title}{use=structure, fg=textcolor, bg=structure.fg}
\setbeamercolor{block body}{
	parent=normal text, use=block title, bg=block title.bg!10}
%\setbeamercolor{block title}{bg=uclablue!100, fg=white}


\setbeamertemplate{enumerate items}[default]
\setbeamerfont*{itemize/enumerate body}{size=\normalsize}
\setbeamerfont*{itemize/enumerate subbody}{parent=itemize/enumerate body}
\setbeamerfont*{itemize/enumerate subsubbody}{parent=itemize/enumerate body}

\setbeamercovered{dynamic}
\setbeamercovered{invisible}
\beamertemplatenavigationsymbolsempty

% Insert table of content frames between sections automatically. See http://en.wikibooks.org/wiki/LaTeX/Presentations
\AtBeginSection[]
{
  \begin{frame}
    \frametitle{Table of Contents}
    \tableofcontents[currentsection]
    %    \tableofcontents[currentsection, currentsubsection]
  \end{frame}
}

\title[Bayesian Sparse GLM]{Practical introduction to Hamiltonian Monte Carlo}
\author{Aki Nishimura\inst{1}}
\institute[]{Department of Biomathematics, University of California - Los Angeles \inst{1}}
\date{\today}


\begin{document}
\frame{\titlepage}

\section{Title TBD}

\frame{
	\frametitle{Set-up and terminologies for HMC}	
	\begin{itemize}
		\item To sample from the distribution of interest $\pi_\Position(\bposition)$, HMC augments the parameter space with an auxiliary parameter $\bmomentum \in \mathbb{R}^\nparam \sim \normal(\mathbf{0}, m \I)$.
		\item HMC then samples from the joint density
		\begin{equation*}
		\pi(\bposition, \bmomentum) 
			\propto \pi_\Position(\bposition) \times \mathcal{N}\left( \bmomentum \, ; \mathbf{0}, m \I \right) 
		\end{equation*}
		where $\bposition$ and $\bmomentum$ are referred to as \textit{position} and \textit{momentum} variables.
		\item HMC terminologies:
		\begin{itemize}
			\item $U(\bposition) = - \log \pi_\Theta(\bposition)$ : \textit{potential energy}.
			\item $K(\bmomentum) = \frac{1}{m} \bmomentum^\transpose \bmomentum$ : \textit{kinetic energy}.
			\item $H(\bposition, \bmomentum) = U(\bposition) + K(\bmomentum) = - \log \pi(\bposition, \bmomentum)$ : \newline \phantom{$H(\bposition, \bmomentum) = U(\bposition) + K(\bmomentum)$ =} \textit{Hamiltonian} or \textit{total energy}.
		\end{itemize}
	\end{itemize}
}

\frame{
	\frametitle{Transition kernel of HMC}
	\begin{itemize}
	\item HMC generates a proposal by combining momentum randomization and \textit{deterministic} transition of simulated Hamiltonian dynamics.
	\end{itemize}
	\begin{block}{HMC transition rule}
		Given the current state $(\bposition_0, \bmomentum_0)$, HMC generates the next state as follows:
		\begin{enumerate}
			\item Randomize momentum: $\bmomentum_0 \sim \mathcal{N}\left(\mathbf{0}, m \I \right)$.
			\item Generate a proposal $(\bposition^*, \bmomentum^*) \approx (\bposition(\integrationTime), - \bmomentum(\integrationTime))$ by approximating \newline the solution $\left\{ (\bposition(t), \bmomentum(t)) : t \in [0, \integrationTime] \right\}$ of \textit{Hamilton's equation}
				\vspace{-.3\baselineskip}
				\begin{equation} \label{eq:hamilton}
				\vspace{-.3\baselineskip}
				\begin{aligned}
				\frac{{\rm d} \bposition}{{\rm d} t} 
				&= m^{-1} \bmomentum, \quad
				\frac{{\rm d} \bmomentum}{{\rm d} t} 
				= - \nabla_{\bposition} U(\bposition) 
				\end{aligned}
				\end{equation}
				with initial condition $(\bposition(0), \bmomentum(0)) = (\bposition_0, \bmomentum_0)$.
			\item Accept or reject the proposal. 
				% with the probability $$ \min\left\{1, \pi(\bposition^*, \bmomentum^*) / \pi(\bposition, \bmomentum) \right\} $$
		\end{enumerate}
	\end{block}
}

\end{document}